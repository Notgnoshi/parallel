\documentclass{article}

% !TEX root = project.tex

\usepackage{afterpage}
\usepackage{algorithm}
\usepackage{algpseudocode}
\usepackage{amsmath}
\usepackage{amssymb}
\usepackage{amsthm}
\usepackage[titletoc]{appendix}
\usepackage{array}
\usepackage[english]{babel}
\usepackage{booktabs}
\usepackage{cancel}
\usepackage{color}
\usepackage{eqparbox}
\usepackage{float}
\usepackage[margin=1in]{geometry}
\usepackage{graphicx}
\usepackage[hidelinks]{hyperref}
\usepackage[utf8]{inputenc}
\usepackage{lipsum}
\usepackage[cache=false]{minted}
\usepackage{parskip}
\usepackage{pgfplots}
\usepackage{scalerel}
\usepackage{skull}
\usepackage{subcaption}
\usepackage{titling}
\usepackage{textcomp}
\usepackage{tikz}
\usepackage{tikz-3dplot}
\usepackage{titlesec}
\usepackage{textcomp}
\usepackage[nottoc]{tocbibind}
\usepackage[textsize=small]{todonotes}
\usepackage[normalem]{ulem}

% Document Settings

\definecolor{__minted_background_color}{rgb}{0.95, 0.95, 0.98}
\definecolor{__minted_highlight_color}{rgb}{0.88, 0.88, 1.0}
\setminted{autogobble=true,
           style=tango,
           breaklines,
           bgcolor=__minted_background_color,
           highlightcolor=__minted_highlight_color,
           % Math escape breaks the preprocessor #pragmas...
           % mathescape, % Escape math mode everywhere.
           % texcomments,  % Enable latex code inside of comments. Useful for referencing equations.
    }

\usetikzlibrary{arrows, shapes, positioning}
\pgfplotsset{compat=1.15}
\numberwithin{equation}{section}
% Sets the width of the margin TODO notes
\setlength{\marginparwidth}{0.84in}
\reversemarginpar{}

% All I want is to have comment italicized, but I cant figure out how
% to properly modify the existing \Comment macro.
% \algrenewcomment[1]{\hfill\eqparbox{COMMENT}{\textit{// #1}}}
\algnewcommand{\IComment}[1]{\Comment{\textit{#1}}}

\graphicspath{{./figures/}}

% Make \paragraph{}s end with newlines and no indents
\titleformat{\paragraph}
    {\normalfont\bfseries}
    {}
    {0pt}
    {}
% Make \paragraph{}s not have a space between the title and the text
\titlespacing*{\paragraph}{0pt}{1ex}{-\parskip}

% Document Definitions

\newcommand{\C}{\mathbb{C}}
\newcommand{\R}{\mathbb{R}}
\newcommand{\Z}{\mathbb{Z}}
\renewcommand{\O}{\mathcal{O}}

\theoremstyle{definition}
\newtheorem{defn}{Definition}[section]

\theoremstyle{plain}
\newtheorem{thm}{Theorem}[section]

\renewcommand{\qedsymbol}{$\blacksquare$}

% An inline TODO command. Doesn't play nicely with \todotableofcontents
\newcommand\todoinline[2][]{\todo[inline, caption={TODO}, #1]{
\begin{minipage}{\textwidth-4pt}#2\end{minipage}}}

% Draw clouds around things. Useful in mathematical proofs.
\newcommand{\cloud}[4][\dots]{
    \raisebox{-0.4\height}{
        \begin{tikzpicture}
            \node [cloud,
                   draw,
                   cloud puffs=#2,
                   cloud ignores aspect,
                   minimum height=#3,
                   minimum width=#4] {#1};
        \end{tikzpicture}
    }
}


\title{Homework 1}
\author{Austin Gill}

\begin{document}
\maketitle

\section{}
    \begin{quote}
        Recall that OpenMP creates private variables for reduction variables, and these private variables are initialized to the identity element for the reduction operator. For example, if the operator is addition, the private variables are initialized to 0, while if the operator is multiplication, the private variables are initialized to 1.

        What are the identity values for these operators: \mintinline{c}{&&}, \mintinline{c}{||}, \mintinline{c}{|}, and \mintinline{c}{^}?
    \end{quote}

\section{}
    \begin{quote}
        Suppose OpenMP did not have the \mintinline{c}{reduction} clause. Show how to implement an \textit{efficient} parallel reduction by adding a private variable and using the \mintinline{c}{critical} pragma.
    \end{quote}

\section{}
    \begin{quote}
        For each of the following code segments, use OpenMP pragmas to make the loop parallel, or explain why the code segment is not suitable for parallel execution.

        \begin{enumerate}
            \item \begin{minted}{c}
                for(i = 0; i < (int) sqrt(x); i++)
                {
                    a[i] = 2.3 * i;
                    if(i < 10)
                        b[i] = a[i];
                }
            \end{minted}

            \item \begin{minted}{c}
                flag = 0;
                for(i = 0; i < n && !flag; i++)
                {
                    a[i] = 2.3 * i;
                    if(a[i] < b[i])
                        flag = 1;
                }
            \end{minted}

            \item \begin{minted}{c}
                for(i = 0; i < n; i++)
                {
                    a[i] = foo(i);
                }
            \end{minted}

            \item \begin{minted}{c}
                for(i = 0; i < n; i++)
                {
                    a[i] = foo(i);
                    if(a[i] > b[i])
                        a[i] = b[i];
                }
            \end{minted}

            \item \begin{minted}{c}
                for(i = 0; i < n; i++)
                {
                    a[i] = foo(i);
                    if(a[i] < b[i])
                        break;
                }
            \end{minted}

            \item \begin{minted}{c}
                dotp = 0;
                for(i = 0; i < n; i++)
                {
                    dotp += a[i] * b[i];
                }
            \end{minted}

            \item \begin{minted}{c}
                for(i = k; i < 2 * k; i++)
                {
                    a[i] = a[i] + a[i-k];
                }
            \end{minted}

            \item \begin{minted}{c}
                for(i = k; i < n; i++)
                {
                    a[i] = b * a[i - k];
                }
            \end{minted}
        \end{enumerate}
    \end{quote}

\section{}
    \begin{quote}
        Given a task that can be divided into $m$ subtasks, each requiring one unit of time, how much time is needed for an $m$-stage pipeline to process $n$ tasks?
    \end{quote}

\section{}
    \begin{quote}
        If the address of the nodes in a hypercube has $n$ bits, at most how many nodes can there be, and how many edges does each node have?

        Give an algorithm that routes a message from node $u$ to node $v$ in this $k$-node hypercube in no more than $\log k$ steps.
    \end{quote}

\section{}
    \begin{quote}
        Research the \textit{shuffle-exchange} network topology. Draw the network with 16 processor nodes (numbering each node in binary, showing shuffle and exchange links). If there are $k$ bits in the address, how many nodes are there? With $n$ nodes, what is the diameter and bisection width of the network? How many edges per node are there?
    \end{quote}

\end{document}
